\documentclass{article}
\usepackage{graphicx} % Required for inserting images

\title{File Transfer Protocol}
\author{Do Tuan Dung BI12-104}
\date{\today}

\begin{document}

\maketitle

\begin{abstract}
This report documents the implementation of a file transfer system using sockets in C. The system enables a client to send a file to a server over a network connection. The report details the system overview, implementation details of the server and client code, and provides relevant code snippets for illustration.
\end{abstract}

\section{Introduction}

This report describes the development of a file transfer system using sockets in C. The project allows a client to send a file to a server on the same network, facilitating data exchange between machines.

\section{System Overview:}
1. Server-Side Design:
\\ +) Rank Restriction: Only one process acts as the server (rank 0) to avoid conflicts.
\\+) Access Control: The provided code includes a basic access control check using access to ensure read permissions on the requested file. This can be further enhanced with user authentication and authorization mechanisms.
\\ +) File Type Check: The code checks if the requested file is a regular file to prevent accidental transfer of directories or special files.
\\+) File Transfer Loop: The server enters a loop, waiting for filenames from clients and sending the requested file data in chunks (using MPI Send).

2. Client-Side Design:
\\+) Rank Check: The code assumes only one process acts as the client (rank 0) for simplicity. You could modify it to allow multiple clients to send requests concurrently.
\\+) Filename Input: The client prompts the user for a filename and sends it to the server.
\\+) File Size (Optional): The client can optionally receive the file size from the server for progress tracking.
\\+) Data Reception: The client receives file data in chunks from the server using MPI_Recv and writes it to an output file.
\subsection{MPI Client:}
The client establishes a connection, sends the filename, and transmits the file data. Here is the "mpiclient.c" file:
\begin{verbatim}
#include <mpi.h>
#include <stdio.h>
#include <stdlib.h>
#include <string.h>

#define BUFFER_SIZE 1024

int main(int argc, char *argv[]) {
  MPI_Init(&argc, &argv);

  int rank, size;
  MPI_Comm_rank(MPI_COMM_WORLD, &rank);
  MPI_Comm_size(MPI_COMM_WORLD, &size);

  if (rank == 0) {
    printf("Client (rank %d): Enter the filename to transfer: ", rank);
    char filename[BUFFER_SIZE];
    fgets(filename, BUFFER_SIZE, stdin);
    filename[strcspn(filename, "\n")] = 0; 

    MPI_Send(filename, BUFFER_SIZE, MPI_CHAR, 0, 0, MPI_COMM_WORLD);

    int file_size;
    MPI_Recv(&file_size, 1, MPI_INT, 0, 0, MPI_COMM_WORLD, MPI_STATUS_IGNORE);
    printf("Server: File size: %d bytes\n", file_size); 

    int bytes_received;
    char buffer[BUFFER_SIZE];
    FILE *output_file = fopen("received_file.txt", "wb");  
    while ((bytes_received = MPI_Recv(buffer, BUFFER_SIZE, MPI_CHAR, 0, 0, MPI_COMM_WORLD, MPI_STATUS_IGNORE)) > 0) {
      fwrite(buffer, sizeof(char), bytes_received, output_file);
    }

    fclose(output_file);
    printf("Client: File transfer complete.\n");
  }

  MPI_Finalize();
  return 0;
}

\end{verbatim}

\subsection{MPI Server:}
The server awaits client connections and accepts file transfers. Here is the "mpiserver.c" file:
\begin{verbatim}
#include <mpi.h>
#include <stdio.h>
#include <stdlib.h>
#include <string.h>
#include <unistd.h>  
#include <sys/stat.h>  

#define BUFFER_SIZE 1024

int main(int argc, char *argv[]) {
  MPI_Init(&argc, &argv);

  int rank, size;
  MPI_Comm_rank(MPI_COMM_WORLD, &rank);
  MPI_Comm_size(MPI_COMM_WORLD, &size);

  if (rank != 0) {
    MPI_Finalize();
    return 0;
  }

  char filename[BUFFER_SIZE];
  int file_fd;
  struct stat file_stat;

  while (1) {
    MPI_Recv(filename, BUFFER_SIZE, MPI_CHAR, MPI_ANY_SOURCE, 0, MPI_COMM_WORLD, MPI_STATUS_IGNORE);
    if (access(filename, R_OK) != 0) {
      printf("Server: Access denied for '%s'\n", filename);
      continue;
    }

    if (stat(filename, &file_stat) < 0 || !S_ISREG(file_stat.st_mode)) {
      printf("Server: '%s' is not a regular file.\n", filename);
      continue;
    }

    file_fd = open(filename, O_RDONLY);
    if (file_fd < 0) {
      printf("Server: Error opening file '%s'\n", filename);
      continue;
    }

    int file_size = file_stat.st_size;
    MPI_Send(&file_size, 1, MPI_INT, MPI_ANY_SOURCE, 0, MPI_COMM_WORLD);

    int bytes_read;
    char buffer[BUFFER_SIZE];
    while ((bytes_read = read(file_fd, buffer, BUFFER_SIZE)) > 0) {
      MPI_Send(buffer, bytes_read, MPI_CHAR, MPI_ANY_SOURCE, 0, MPI_COMM_WORLD);
    }

    close(file_fd);
  }

  MPI_Finalize();
  return 0;
}


\end{verbatim}

\end{document}

